
\subsection{Monochromatic filters}
Let $W \in \mathbb{R}^{C \times X \times Y \times F}$ denote the first convolutional layer weights of a trained network. The number of input channels, $C$, is 3 and each channel corresponds to a different color component (either RGB or YUV). We have found that the color components of weights from a trained convolutional neural network have low dimensional structure. In particular, the weights can be well approximated by projecting the color dimension down into a 1D subspace. Figure ?? shows the original first layer convolutional weights of a trained network and the weights after the color dimension has been projected into 1D lines. 

The approximation is computed as follows. First, for every output feature, $f$, we consider consider the matrix $W_f \in \mathbb{R}^{C \times XY }$, where the spatial dimensions have been combined, and find the singular value decomposition, 
\begin{equation*}
	W_f = U_f S_f V_f^{\top}
\end{equation*}
where $U_f \in \mathbb{R}^{C \times C}, S_f \in \mathbb{R}^{C \times XY}, V_f \in \mathbb{R}^{XY \times XY}$. We then take the rank-1 approximation to $W_f$ 
\begin{equation*}
	\tilde{W}_f = \tilde{U}_f \tilde{S}_f \tilde{V}_f^{\top}
\end{equation*}
where $\tilde{U}_f \in \mathbb{R}^{C \times 1}, \tilde{S}_f \in \mathbb{R}, \tilde{V}_f \in \mathbb{R}^{1 \times XY}$.

This approximation corresponds to shifting from $C$ color channels to 1 color channel for each output feature. We can further exploit the regularity in the weights by sharing the color component basis between different output features. We do this by clustering the $F$ left singular vectors,  $\tilde{U}_f$, of each output feature $f$ into $C'$ equal sized clusters, where $C'$ is much smaller than $F$. Then, for each of the $\frac{F}{C'}$ output feature, $f$, that is assigned to cluster $c$, we can approximate $W_f$ with
\begin{equation*}
	\tilde{W}_f = U_c \tilde{S}_f \tilde{V}_f^{\top}
\end{equation*}
where $U_c \in \mathbb{R}^{C \times 1}$ is the cluster center for cluster $c$ and $\tilde{S}_f$ and $\tilde{V}_f$ as as before. 

This low-rank approximation allows for a more efficient computation of the convolutional layer output. By decomposing the approximated weights into two tensors. Let $W_C \in \mathbb{R}^{C' \times C}$ denote the color transform matrix where the rows of $W_c$ are the cluster centers $U_c^{\top}$. Let $W_{mono} \in \mathbb{R}^{X \times Y \times F}$ denote the monochromatic weight tensor containing $ \tilde{S}_f \tilde{V}_f^{\top}$ for each of the $F$ output features. Given this decomposition, we can compute the output of the convolutional layer by first transforming the input signal, $I \in \mathbb{R}^{C \times N \times M}$ into a different basis using the color transform matrix: $\tilde{I} = W_c $

By decomposing the approximated weights into two tensors, this low-rank approximation allows for a more efficient computation of the convolutional layer output. Let $W_C \in \mathbb{R}^{C' \times C}$ denote the color transform matrix where the rows of $W_c$ are the cluster centers $U_c^{\top}$. Let $W_{mono} \in \mathbb{R}^{X \times Y \times F}$ denote the monochromatic weight tensor containing $ \tilde{S}_f \tilde{V}_f^{\top}$ for each of the $F$ output features. Given this decomposition, we can compute the output of the convolutional layer by first transforming the input signal, $I \in \mathbb{R}^{C \times N \times M}$ into a different basis using the color transform matrix: 
\begin{equation*}
	\tilde{I} = W_c \otimes I
\end{equation*}
where $\tilde{I} \in \mathbb{R}^{C' \times N \times M}$. After the color transformation, each of the $f$ filters in $W_{mono}$ is monochromatic in the sense that it only acts upon one of the $C'$ color channels. Concretely, target value, $\tilde{T}$, of the approximated convolutional layer for a particular output feature $f$, and spatial location $(x, y)$ is
\begin{align*}
	\tilde{T}(f, x, y) &= \sum_{x' = 1}^{X} \sum_{y' = 1}^{Y} \Big(\sum_{c = 1}^{C} I(c, x + x', y + y') W_c(c', c) \Big) \\
			&\hspace{3mm} \times (c, x + x', y + y') W(x', y', f) \\
			&= \sum_{x' = 1}^{X} \sum_{y' = 1}^{Y} \tilde{I}(c', x + x', y + y') W(x', y', f)
\end{align*} 
where $c' \in [1, ..., C']$ is the cluster feature $f$ is assigned to. If the color transformation is computed once at the outset, then the number of operations performed is significantly reduced.  

\subsubsection{Complexity analysis}

\subsubsection{Empirical performance}

