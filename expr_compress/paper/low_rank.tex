\section{Low Rank Approximations}
In this section, we give theoretical background on low rank approximations. First, we discuss simplest setting, which is
for matrices (two dimensional tensors). We then consider the approximation of 4-dimensional tensors of convolution weights.


\subsection{Matrix Low Rank Approximation}
Let $X \in \mathbb{R}^{n \times m}$ denote the input to a fully connected layer of a neural network and let $W \in \mathbb{R}^{m \times k}$ denote the weight matrix for the layer. Matrix multiplication,  the main operation for fully connected layers, costs $O(nmk)$. However, $W$ is likely to have a low-rank structure and thus have several eigenvalues close to zero. These dimensions can be interpreted as noise, and thus can be eliminated without harming the accuracy of the network. We now show how to exploit this low-rank structure and to $XW$ much faster than $O(nmk)$. 


Every matrix $W \in \mathbb{R}^{m \times k}$ can be expressed using singular value decomposition:
\begin{equation*}
	W = USV^{\top}\text{, where }U \in \mathbb{R}^{m \times m}, S \in \mathbb{R}^{m \times k}, V \in \mathbb{R}^{k \times k}
\end{equation*}
$S$ is has eigenvalues on the diagonal, and zeros elsewhere. $W$ can be approximated by choosing the $t$ largest 
eigenvalues from $S$. We can write the approximation as
\begin{equation*}
	\tilde{W} = \tilde{U}\tilde{S}\tilde{V}^{\top}\text{, where }\tilde{U} \in \mathbb{R}^{m \times t}, \tilde{S} \in \mathbb{R}^{t \times t}, \tilde{V} \in \mathbb{R}^{t \times k}
\end{equation*}

Now the computation $X\tilde{W}$ can be done in $O(nmt + nt^2 + ntk)$, which, for sufficiently small $t$ can be significantly smaller than $O(nmk)$. 

\subsection{Tensor Low Rank Approximations}

For a 3-tensor, $M \in \mathbb{R}^{n \times m \times k}$, we can construct a rank 1 approximation by finding a decomposition that minimizes 
\begin{equation*}
	\| M - \alpha \otimes \beta \otimes \gamma \|_F
\end{equation*} 
where $\alpha \in \mathbb{R}^n$, $\beta \in \mathbb{R}^m$, $\gamma \in \mathbb{R}^m$ and $\|X\|_F$ denotes the Frobenius norm.

This easily extends to a rank K approximation using a greedy algorithm: First find the best rank 1 approximation and then iteratively find the best rank 1 approximation to the remaining tensor after subtracting the rank 1 approximation found in the previous step. %Ugliest sentence ever, fix! %
 The rank K approximation is given as a sum of the rank 1 tensors  
\begin{equation*}
	\tilde{M} = \sum_{k = 1}^{K} \alpha_k \otimes \beta_k \otimes \gamma_k 
\end{equation*} 

\subsection{Low Rank Approximations of Convolutional Filters}

In typical object recognition architectures, the weights of convolutional layers at the end of training exhibit strong redundancy and regularity across all dimensions. A particularly simple way to exploit such regularity is to 
linearly compress the tensors, which amounts to finding low-rank approximations.

Convolution weights can be described as a $4$-dimensional tensor. Let $W \in \mathbb{R}^{C \times X \times Y \times F}$ 
denote such a weight tensor. $C$ is the number of number of input channels, $X$ and $Y$ are the special dimensions of the kernel, and $F$ is the target number of feature maps.
Let $I \in \mathbb{R}^{C \times N \times M}$ denote an input signal where $C$ is the number of input maps, and $N$ and $M$ are the spatial dimensions of the maps.
The target value, $T = I \ast W$, of a generic convolutional layer for a particular output feature, $f$, and spatial location, $(x, y)$, is defined as
\begin{align*}
\label{convlayereq}
&T(f,x,y) = \\
&\sum_{c=1}^C \sum_{x'=1}^{X} \sum_{y'=1}^{Y} I(c,x+x',y+y') W(c,x',y',f)
\end{align*}

We show how to approximate $W$ with a low rank tensor that allows for a more efficient computation of the convolution. The approximations will be more efficient in two senses: both the number of floating point operations required to compute the convolution output and the number of parameters that need to be stored will be dramatically reduced. 

The first convolutional layer in the standard architecture receives three color channels, typically in RGB or YUV space, as input whereas later hidden layers typically receive a much larger number of feature maps that have resulted from computations performed in previous layers. As a result, the first layer weights often have a markedly different structure than the weights in later convolutional layers. We have found that different approximation techniques are well suited to the different layers. The first approach, which we call the monochromatic filter approximation, can be applied to the weights in the first convolutional layer. The second approach, which we call the bi-clustering approximation, can be applied to later convolutional layers where the number of input and output maps is large. 
