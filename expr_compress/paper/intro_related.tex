
\begin{abstract}
We present techniques for speeding up the test-time evaluation of large convolutional networks, designed for object recognition tasks. These models deliver impressive accuracy but each image evaluation requires millions of floating point operations, making their deployment on smartphones and Internet-scale clusters problematic. The computation is dominated by the convolution operations in the lower layers of the model. We exploit the linear structure present within the 
convolutional filters to derive approximatations that significantly reduce the required computation. Using large state-of-the-art models, we demonstrate speedups by a factor of $2-4$x, while keeping the accuracy within $1\%$ of the original model. 
\end{abstract}

\section{Introduction}

Large neural networks have recently demonstrated impressive
performance on a range of speech and vision tasks. However the size of
these models can make their deployment at test time problematic. For
example, mobile computing platforms are limited in their CPU speed,
memory and battery life. At the other end of the spectrum,
Internet-scale deployment of these models requires thousands of
servers to process the 100's of millions of images per day. The
electrical and cooling costs of these servers required is significant.

Training large neural networks (NN) can take weeks, or even
months. This hinders research and consequently there have been
extensive efforts devoted to speeding up training procedure.  However,
there are relatively few efforts are improving the {\em test-time}
performance of the models. 

In this paper we focus on speeding up the
evaluation of {\em trained} networks, without compromising
performance. We consider convolutional neural networks used for
computer vision tasks, since they are large and widely used in
commercial applications. Within these models, most of the time ($\sim90\%$) is spent in the
convolution operations in the lower layers of the model. The remaining
operations: pooling, contrast normalization and the upper
fully-connected layers collectively take up the remaning $10\%$.

We present two novel methods for speeding up the convolution
operations. One involves projecting the input image into a set of 1-D
color sub-spaces. This allows the filters in the first of layer of the
model to be monochromatic (i.e. reducing the color channels from three
to one), thereby saving a factor of 3 in computation. The second
approach, applied to subsequent convolution layers, involves
clustering the filters into a set of low-dimensional linear
sub-spaces, each of which is represented by a set of tensor
outer-products. Collectively, our techniques speed up execution by
factor of $2-4$ while keeping prediction accuracy within $1\%$ of the
original model. These gains allow the use of larger, higher
performance models than would otherwise be practical.


% resource-wise from perspective of companies executing neural networks on internet-scale
% data (e.g. annotating images), this is not the main cost. Major cost is in the
% final stage, where network is evaluated on the target data, which is present in quantities of billions.
% We focus here on speeding up evaluation of \emph{trained} NN, which directly
% maps to the cost of executing NN on internet-scale data.

% % XXX: Maybe we will speak also about real time application.


% We focus in this work on convolutional neural networks used for computer vision tasks. Most of
% computation time during evaluation is spend on convolutional layers i.e. $\sim90\% - 95\%$, while it takes only
% the small fraction of time $\sim 5\%-10\%$ to evaluate rest of layers (pooling, local contrast normalization,
% fully connected). It is worth to note, that most of learnable parameters are kept in fully connected layers $\sim 90\% - 95\%$
% , and convolutional layers constitutes of very small fraction of parameters $\sim 5\% - 10\%$.


% We achieve forward pass speed up by constructing approximations to the convolutional layer kernel. Convolutional kernel
% is a $4$-dimensional tensor, with two spacial dimensions, and two feature maps-to-feature maps dimensions. Kernel of trained
% network has a lot of redundancies in parameters, which we exploit to speed up forward pass, while mildly training off
% prediction accuracy (approximated kernels give prediction within $\sim 1\%$ of the original prediction).


\section{Related Work}
%There have been extensive research devoted speeding up forward pass of neural network. 
%There are few different pathways, how the speed up can be achieved. 


\cite{vanhoucke2011improving} explored the
properties of CPUs to speed up execution.  They present many solutions
specific to Intel and AMD CPUs, however some of their techniques are
general enough to be used for any type of processor.  They describe
how to align memory, and use SIMD operations (vectorized operations on
CPU) to boost the efficiency of matrix multiplication.  Additionally, they
propose the linear quantization of the network weights and input. This
involves representing weights as 8-bit integers (range
$[-1287 128]$), rather than 32-bit floats. This approximation is
similar in spirit to our approach, but differs in that it is applied
to each weight element independently. By contrast, our approximation approach models
the structure within each filter. Potentially, the two approaches
could be used in conjunction. 

% combinIt can also potentially be to 
% approximates the ion, it does so based on the structure of the weights.  each which
% replaces kernel $W$ with few other operations, which give result
% approximately equal to convolution with $W$. Moreover, linear
% quantization can be used in conjunction with methods presented in this
% paper.



The most expensive operations in convolutional networks are the
convolutions in the first few layers. The complexity of this operation
is linear in the area of the receptive field of the filters, which is
relatively large for these layers.  However, \cite{mathieu2013fast} have shown that convolution can be
efficiently computed in Fourier domain, where it becomes element-wise
multiplication (and there is no cost associated with size of receptive
field). They report a forward-pass speed up of around $10x$ (depending on
the kernel size, number of features etc.).  Importantly, this method can
be used jointly with most of techniques presented in this paper.

The use of low-rank approximations in our approach is inspired by work
of \cite{denil2013predicting} who demonstrate the redundancies in neural
network parameters. They show that the weights within a layer can be
accurately predicted from a small (e.g. $\sim 5\%$) subset of them. This
indicates that neural networks are heavily over-parametrized.  All the
methods presented here focus on exploiting the linear structure of this
over-parametrization.

\section{Framework}
We utilize in our studies models trained on Imagenet 2012 dataset. One is developed
by \cite{sermanet2013overfeat}, and other by \cite{zeiler2013visualizing}.
We refer to them respectively as PierreNet, and MattNet. We evaluate our networks on Macbook pro with 2.4 GHz i7 Intel processor, and
all our evaluation code is in implemented in C++ using Eigen3 library \cite{eigenweb} compiled with Intel MKL. We use Intel's implementation of openmp, and multithreading. 
